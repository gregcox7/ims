\chapter*{{\color{oiB}Preface}}
%\chaptertext{}
%\sectiontext{}

\noindent%
OpenIntro Statistics covers a first course in statistics,
providing a rigorous introduction to applied statistics
that is clear, concise, and accessible.
This book was written with the undergraduate level in mind,
but it's also popular in high schools and graduate courses.
\vspace{3mm}

We hope readers will take away three ideas from
this book in addition to forming a foundation of statistical
thinking and methods.\vspace{-1mm}
\begin{itemize}
\setlength{\itemsep}{0mm}
\item
    Statistics is an applied field with a wide range
    of practical applications.
\item
    You don't have to be a math guru to learn
    from real, interesting data.
\item
    Data are messy, and statistical tools are imperfect.
    But, when you understand the strengths and weaknesses of
    these tools, you can use them to learn about the world.
\end{itemize}


%\subsection*{Is this a data science book?}
%
%\noindent%
%Short answer: yes.
%Long answer: it depends what you mean by \term{data science},
%since two types of data scientists have emerged.
%\vspace{3mm}
%
%\noindent%
%Type~A data scientists focus on \emph{analysis},
%such as exploratory data analysis, inference,
%model building, and other related topics.
%Type~B data scientists focus on \emph{building},
%typically in the form of machine learning models
%or other systems.
%As you might expect, these two types share many skills,
%though their main focuses differ.
%This book focuses on skills most commonly used by
%Type~A data scientists.
%For more thoughts, please check out the following page:
%\begin{center}
%\oiRedirect{data_science_types}{{\color{red}BROKEN}}
%\end{center}
%\vspace{3mm}
%
%\noindent%


\subsection*{{\color{oiB}Textbook overview}}

\noindent%
The chapters of this book are as follows:%\vspace{2mm}
\begin{description}
\setlength{\itemsep}{0mm}
\item[1. Introduction to data.]
    Data structures, variables,
    and basic data collection techniques.
\item[2. Summarizing data.]
    Data summaries, graphics,
    and a teaser of inference using randomization.
\item[3. Probability.]
    Basic principles of probability.
    %This chapter is not required for the later chapters.
\item[4. Distributions of random variables.]
    The normal model and other key distributions.
\item[5. Foundations for inference.]
    %Introduction to uncertainty in point estimates,
    %confidence intervals, and hypothesis tests.
    General ideas for statistical inference in the context
    of estimating the population proportion.
\item[6. Inference for categorical data.]
    Inference for proportions and tables using the normal
    and chi-square distributions.
\item[7. Inference for numerical data.]
    Inference for one or two sample means using the
    \mbox{$t$-distribution},
    statistical power for comparing two groups,
    and also comparisons of many
    means using ANOVA.
\item[8. Introduction to linear regression.]
    Regression for a numerical outcome with one predictor variable.
    Most of this chapter could be covered after
    Chapter~\ref{introductionToData}.
\item[9. Multiple and logistic regression.]
    Regression for numerical and categorical data
    using many predictors. %for an accelerated course.
\end{description}


%\newpage

\noindent%
\emph{OpenIntro Statistics} supports flexibility
in choosing and ordering topics.
If the main goal is to reach multiple regression
(Chapter~\ref{ch_regr_mult_and_log})
as quickly as possible, then the following are the
ideal prerequisites:
\begin{itemize}
\setlength{\itemsep}{0mm}
\item Chapter~\ref{ch_intro_to_data},
    Sections~\ref{numericalData},
    and Section~\ref{categoricalData} for a solid
    introduction to data structures and statistical
    summaries that are used throughout the book.
\item Section~\ref{normalDist}
    for a solid understanding of the normal distribution.
\item Chapter~\ref{ch_foundations_for_inf}
    to establish the core set of inference tools.
%\item Section~\ref{oneSampleMeansWithTDistribution}
%    and Chapter~\ref{ch_regr_simple_linear}
%    provide required for multiple regression with a numerical
%    outcome.
%    For the remaining chapters, they could be tackled in
%    almost any order, with the exception that
%    
%    
%    come before Chapter~\ref{ch_regr_mult_and_log}.
\item Section~\ref{oneSampleMeansWithTDistribution}
    to give a foundation for the $t$-distribution
\item Chapter~\ref{ch_regr_simple_linear}
    for establishing ideas and principles for single
    predictor regression.
%    introduce the 
%    which introduces the $t$-distribution, should come before
%    Section~\ref{oneSampleMeansWithTDistribution}
%Chapters~\ref{ch_inference_for_props}-\ref{ch_regr_mult_and_log},
%    could be tackled in
%    almost any order, with the exception that
%    Section~\ref{oneSampleMeansWithTDistribution}
%    and Chapter~\ref{ch_regr_simple_linear}
%    come before Chapter~\ref{ch_regr_mult_and_log}.
%\item Sections~\ref{ch_inference_for_props}
%    and~\ref{} are recommended before logistic regression.
\end{itemize}
%One conspicuously missing topic from the list above is the
%chapter on Probability.
%While useful for a deeper understanding of the calculations,
%especially for anyone looking to take a second course in
%statistics, it is not required reading when the focus is on
%applied data analysis.


\subsection*{{\color{oiB}Examples and exercises}}
%, and appendices}

\noindent%
Examples are provided to establish an understanding of how
to apply methods

\begin{examplewrap}
\begin{nexample}{This is an example.
    When a question is asked here, where can the answer be found?}
  The answer can be found here, in the solution section
  of the example!
\end{nexample}
\end{examplewrap}

\noindent%
When we think the reader should be ready to try determining
the solution to an example, we frame it as Guided Practice.

\begin{exercisewrap}
\begin{nexercise}
The reader may check or learn the answer to any Guided Practice
problem by reviewing the full solution in a footnote.\footnotemark{}
%Readers are strongly encouraged to attempt these practice problems.
\end{nexercise}
\end{exercisewrap}
\footnotetext{Guided Practice problems are intended to stretch
  your thinking, and you can check yourself by reviewing the
  footnote solution for any Guided Practice.}

\noindent%
Exercises are also provided at the end of each section
as well as review exercises at the end of each chapter.
Solutions are given for odd-numbered exercises in
Appendix~\ref{eoceSolutions}.
%Probability tables for the normal, $t$,
%and chi-square distributions are in
%Appendix~\ref{distributionTables}.


\subsection*{{\color{oiB}Additional resources}}

Video overviews, slides, statistical software labs,
data sets used in the textbook,
and much more are readily available at\\[-5mm]
\begin{center}
\oiRedirect{os}
    {\color{black}\textbf{openintro.org/os}}
\end{center}
%Data sets for this textbook are available on the website
%and in a companion R package.\footnote{Diez DM,
%    Barr CD, \c{C}etinkaya-Rundel M. 2015.
%    \texttt{openintro}: OpenIntro data sets and supplement
%    functions.
%    \oiRedirect{textbook-github_openintro}
%        {github.com/OpenIntroOrg/openintro-r-package}.}
%All of these resources are free and may be used with
%or without this textbook as a companion.
We also have improved the ability to access data in this book
through the addition of Appendix~\ref{appendix_data},
which provides additional information for each of the data sets
used in the main text and is new in the Fourth Edition.
Online guides to each of these data sets are also provided at
\oiRedirect{data}
    {\color{black}\textbf{openintro.org/data}}
and through a
\oiRedirect{textbook-github_openintro}
    {companion R~package}.
% Official:
% http://www.openintro.org/package/openintro
% Currently redirect it to:
% http://openintrostat.github.io/openintro-r-package/
\vspace{3mm}

\noindent%
We appreciate all feedback as well as reports of any
typos through the website.
A short-link to report a new typo or review known typos is
\oiRedirect{os_typos}
    {\color{black}\textbf{openintro.org/os/typos}}. \vspace{3mm}

\noindent%
For those focused on statistics at the high school level,
consider
\oiRedirect{textbook-books}
    {\emph{Advanced High School Statistics}},
which is a version of \emph{OpenIntro Statistics} that has
been heavily customized by \oiRedirect{people}{Leah Dorazio}
for high school courses and
AP\textsuperscript{\textregistered} Statistics.


\subsection*{{\color{oiB}Acknowledgements}}
This project would not be possible without the passion and
dedication of many more people beyond those on the author list.
The authors would like to thank the
\oiRedirect{textbook-openintro_about}{OpenIntro Staff}
for their involvement and ongoing contributions.
We~are also very grateful to the hundreds of students
and instructors who have provided us with valuable feedback
since we first started posting book content in~2009. \vspace{3mm}

\noindent%
We also want to thank the many teachers who helped review
this edition, including
Laura Acion,
\oiRedirect{matthew_e_aiello-lammens}
    {Matthew E. Aiello-Lammens},
\oiRedirect{jonathan_akin}{Jonathan Akin},
Stacey C. Behrensmeyer,
Juan Gomez,
Jo Hardin,
\oiRedirect{nicholas_horton}{Nicholas Horton},
\oiRedirect{danish_khan}{Danish Khan},
\oiRedirect{peter_hm_klaren}{Peter H.M. Klaren},
Jesse Mostipak,
Jon C. New,
Mario Orsi,
Steve Phelps,
and David Rockoff.
We appreciate all of their feedback, which helped
us tune the text in significant ways and greatly
improved this book.
