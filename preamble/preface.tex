\chapter*{Preface}

%\vspace{-50mm}

%{\Large Preface} \\

This book may be downloaded as a free PDF at \href{http://www.openintro.org}{\color{black}\textbf{openintro.org}}. %This download is free and need not be supplemented by the purchase of a paperback copy.
\vspace{3mm}

\noindent We hope readers will take away three ideas from this book in addition to forming a foundation of statistical thinking and methods.\vspace{-1mm}
\begin{enumerate}
\setlength{\itemsep}{0mm}
\item[(1)] Statistics is an applied field with a wide range of practical applications.
\item[(2)] You don't have to be a math guru to learn from interesting, real data.
\item[(3)] Data are messy, and statistical tools are imperfect. However, when you understand the strengths and weaknesses of these tools, you can use them to learn interesting things about the~world.
\end{enumerate}
%This textbook is supplemented by many free, online resources to help students apply the methods they learn in this textbook and beyond.


\subsection*{Textbook overview}

The chapters of this book are as follows:
\begin{description}
\setlength{\itemsep}{0mm}
\item[1. Introduction to data.] Data structures, variables, summaries, graphics, and basic data collection techniques.
\item[2. Foundations for inference.] Case studies are used to introduce the ideas of statistical inference with randomization and simulations. The content leads into the standard parametric framework, with techniques reinforced in the subsequent chapters.\footnote{Instructors who have used similar approaches in the past may notice the absence of the bootstrap. Our investigation of the bootstrap has shown that there are many misunderstandings about its robustness. For this reason, we postpone the introduction of this technique until Chapter~\ref{inferenceForNumericalData}.} It is also possible to begin with this chapter and introduce tools from Chapter~\ref{introductionToData} as they are~needed.
\item[3. Inference for categorical data.] Inference for proportions using the normal and chi-square distributions, as well as simulation and randomization techniques.
\item[4. Inference for numerical data.] Inference for one or two sample means using the $t$~distribution, and also comparisons of many means using ANOVA. A special section for bootstrapping is provided at the end of the chapter.
\item[5. Introduction to linear regression.] An introduction to regression with two variables. Most of this chapter could be covered immediately after Chapter~\ref{introductionToData}.
\item[6. Multiple and logistic regression.] An introduction to multiple regression and logistic regression for an accelerated course.
\item[Appendix~\ref{probability}. Probability.] An introduction to probability is provided as an optional reference. Exercises and additional probability content may be found in Chapter~2 of \emph{OpenIntro Statistics} at \href{http://www.openintro.org}{\color{black}\textbf{openintro.org}}. Instructor feedback suggests that probability, if discussed, is best introduced at the very start or very end of the course.
\end{description}

%Conspicuously absent is a dedicated introduction to probability, which may be found in Appendix~\ref{probability}, and exercises for probability practice (along with additional content) may be found in Chapter~2 of \emph{OpenIntro Statistics} (\href{http://www.openintro.org}{\color{black}\textbf{openintro.org}}).

\pagebreak

\subsection*{Examples, exercises, and additional appendices}

Examples and guided practice exercises throughout the textbook may be identified by their distinctive bullets:

\begin{example}{Large filled bullets signal the start of an example.}
Full solutions to examples are provided and often include an accompanying table or figure.
 \end{example}

\begin{exercise}
Large empty bullets signal to readers that an exercise has been inserted into the text for additional practice and guidance. Solutions for all guided practice exercises are provided in footnotes.\footnote{Full solutions are located down here in the footnote!}
\end{exercise}

Exercises at the end of each chapter are useful for practice or homework assignments. Many of these questions have multiple parts, and solutions to odd-numbered exercises can be found in Appendix~\ref{eoceSolutions}. %These end-of-chapter exercises are also available online in a public question bank at \textbf{openintro.org}, and the available selection is constantly growing based on teacher contributions. Numbered citations in end-of-chapter exercises may be found in Appendix~B.

Probability tables for the normal, $t$, and chi-square distributions are in Appendix~\ref{distributionTables}, and PDF copies of these tables are also available from \href{http://www.openintro.org}{\color{black}\textbf{openintro.org}}.

\subsection*{OpenIntro, online resources, and getting involved}

OpenIntro is an organization focused on developing free and affordable education materials.
We encourage anyone learning or teaching statistics to visit \href{http://www.openintro.org}{\color{black}\textbf{openintro.org}} and get involved. We also provide many free online resources, including free course software. 
%, or by creating new material. Students can test their knowledge with practice quizzes for each chapter, or try an application of concepts learned using real data. 
Data sets for this textbook are available on the website and through a companion R package.\footnote{Diez DM, Barr CD, \c{C}etinkaya-Rundel M. 2012. \texttt{openintro}: OpenIntro data sets and supplement functions. \urlwofont{http://cran.r-project.org/web/packages/openintro}.} All of these resources are free, and we want to be clear that anyone is welcome to use these online tools and resources with or without this textbook as a companion.

%Teachers can download the source files for this book, labs, data sets, or create their own custom quizzes and problem sets for students to take at \href{http://www.openintro.org}{\textbf{openintro.org}}. 
%Anyone can download a PDF or the source files of this textbook for modifying and sharing at \href{http://www.openintro.org}{\color{black}\textbf{openintro.org}}.

We value your feedback. If there is a part of the project you especially like or think needs improvement, we want to hear from you. You may find our contact information on the title page of this book or on the \href{http://www.openintro.org/about.php}{About} section of \href{http://www.openintro.org}{\color{black}\textbf{openintro.org}}.

\subsection*{Acknowledgements}

This project would not be possible without the dedication and volunteer hours of all those involved. No one has received any monetary compensation from this project, and we hope you will join us in extending a \emph{thank you} to all those who volunteer with OpenIntro.

The authors would especially like to thank Andrew Bray and Meenal Patel for their involvement and contributions to this textbook. We are also grateful to Andrew Bray, Ben Baumer, and David Laffie for providing us with valuable feedback based on their experiences while teaching with this textbook, and to the many teachers, students, and other readers who have helped improve OpenIntro resources through their feedback.

The authors would like to specially thank George Cobb of Mount Holyoke College and Chris Malone of Winona State University. George has spent a good part of his career supporting the use of nonparametric techniques in introductory statistics, and Chris was helpful in discussing practical considerations for the ordering of inference used in this textbook. Thank you, George and Chris!

